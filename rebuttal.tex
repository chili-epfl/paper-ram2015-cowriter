\documentclass{article}
\usepackage[utf8]{inputenc}


\begin{document}

Dear Editor, Dear Reviewers,

\vspace{2em}

We would like to thank both the reviewers and the editor for going through our
manuscript. As a consequence of your comments, the manuscript underwent
several significant changes, that we summarize hereafter.

While the reviews are generally positive (quoting one of the reviewer, ``this
is an excellent model for this kind of research, and the results are
promising''), we certainly take the two main concerns raised from the reviews:

\begin{itemize}
    \item The reporting of the studies is lacking
    \item Possible ethical concerns are not emphasized
\end{itemize}

This revision of our manuscript attempts to address both these concerns.

\section*{Reporting of the studies}

We have clarified our wording to only draw conclusion from factual measured
data or feedback from the parents/teachers/therapists (this is in particular
the case for the case-study 1),

We have added new quantitative data in the reporting of the case study 2, that
show a quantifiable improvement of handwriting. We want however to remain
cautious in our claims: handwriting acquisition is a complex cognitive skill.
While the child demonstrated clear improvements over the course of our
experiment, measuring the lasting effects of the robot-based remediation is
difficult (as many other individual and social factors may have impacted the
child's performance as well).

Finally, we have added the results of one more experiment (conducted during the
summer 2015), that involved 8 children with handwriting deficits. This
experiment focused on measuring how \emph{serious} the children were at
teaching the robot,

\section*{Ethical concerns}

An explicit paragraph has been added to the discussion that states the possible
ethical concerns raised by this work. While the psychological impact of the
mentor-protégé relationship has been studied in the literature, it has been
done \emph{from the perspective of the protégé}. In our situation, the protégé
is the robot, and the child is the mentor: we could not find existing
literature discussing this situation, but would be nevertheless glad to further
expand the discussion if necessary.

\section*{Other comments}

One of the reviewers asked for clarification on a few other points:

\begin{itemize}

\item \emph{Did we include children who had difficulty with the dynamics of
turn-taking? If so, how did the system fare with those children?} In the case
the children did not pro-actively engaged into turn-taking, the facilitator (be
it one of the experimenter or the therapist) would suggest it.

\item \emph{The handwriting/letter generation is sufficiently described, but
how are the corresponding robot gestures generated?} We relied on NaoQI inverse
kinematics solver. This has been added to the text.

\item \emph{Was an interaction in which the child moves the hand of the NAO to
write the letter considered? Why did the authors choose the current interaction
model in which the child demonstrates on a tablet?} The Nao robot does not
feature force/torque sensing, which prevents most of the interesting direct
physical interaction like moving the hand of the robot. With appropriate
hardware, this interaction modality would be interesting to explore, though.

\item \emph{Could the measure of engagement plotted in Figure 7 be confounded
with the physically-active aspect of the hand-writing activity versus the
passive storytelling activity?} We have decided to replace this figure with an
objective metric (with-me-ness) that relies on the focus of attention of the
children to assess to which extend the child is into the task.

\item \emph{The paper would benefit from further explanation of the adult's
active role throughout the robot-child interaction} The manuscript hopefully
clarifies the role of the facilitator in the different studies.

\end{itemize}

\end{document}
